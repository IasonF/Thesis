% arara: pdflatex: {synctex: yes, action: nonstopmode}
% arara: pdflatex 
% arara: bibtex 
% arara: pdflatex
% arara: pdflatex
% arara: nomencl
% arara: pdflatex: {synctex: yes, action: nonstopmode}

%\documentclass[a4crop]{ntnuthesis} 
\documentclass{ntnuthesis}  
\usepackage{amsmath} %equations and spacings
\usepackage{lipsum} 
\usepackage{graphicx}
\usepackage[small,bf]{caption}
\usepackage[labelformat=simple]{subcaption}
\renewcommand\thesubfigure{(\alph{subfigure})} % see subcaption doc

%spaces under sections
\usepackage[compact]{titlesec}
\titlespacing{\section}{0pt}{*0}{*0}
\titlespacing{\subsection}{0pt}{*0}{*0}
\titlespacing{\subsubsection}{0pt}{*0}{*0}

%Abbreviations or Acronyms
%\usepackage[intoc]{nomencl}
\usepackage{nomencl}
\renewcommand{\nomname}{List of Abbreviations}
\makenomenclature

%Bibliography
%\usepackage{url}
\usepackage{natbib}%\usepackage[sectionbib]{natbib}
\usepackage{chapterbib}
%\bibpunct[:]{(}{)}{;}{a}{}{,} %citation structure
\bibpunct{(}{)}{,}{a}{}{;} 
%\bibpunct{[}{]}{,}{a}{}{;}
%\bibpunct{(}{)}{;}{a}{}{,} % to follow the A&A style
%\usepackage{chapterbib}
\usepackage{hyperref}
%for blank pages:
\usepackage{afterpage}
%\hypersetup{colorlinks=true,citecolor=blue}
\hypersetup{colorlinks=true,citecolor=blue,linkcolor=blue,urlcolor=blue}

%To change box color around the links and citations, you have these other options :
%\hypersetup{citebordercolor=Violet,filebordercolor=Red,linkbordercolor=Blue}

\newcommand{\cmt}[1]{}%inline comment
  
%tabularx
\usepackage{tabularx}
\usepackage{tabulary}
\usepackage{longtable} %table on several pages
\usepackage{booktabs}
\usepackage{multirow}
\usepackage{dcolumn}

%graphics
\usepackage{epstopdf} %support for eps.
\DeclareGraphicsExtensions{.eps,.ps,.pdf,.png,.jpg}
\usepackage{float} % figure placing [H]

%landscape Option
\usepackage{lscape} %Left down
%usepackage{pdflscape} %Left up)

\usepackage[draft]{todonotes}   % notes showed (JJUNJU)
% Select what to do with command \comment:  
% \newcommand{\comment}[1]{}  %comment not showed
\newcommand{\comment}[1]
{\par {\bfseries \color{red} #1 \par}} %comment showed

%Section Numbering and TOC depth
\setcounter{secnumdepth}{2}
\setcounter{tocdepth}{3}

%chapter biblio
\usepackage{chapterbib}

%To generate list of symbols
\newcommand{\addsymbol}[3]{%
  \symboldisplay{#1}{#2}\\%
  \setelem{#3}{#1}
}
\newcommand{\symboldisplay}[2]{%
  $#1$ \parbox{5in}{\dotfill #2}%
  %$#1$ \parbox{5in}{ #2}%
}
%\def\setelem#1{\expandafter\def\csname myarray(#1)\endcsname}
\def\setelem#1{\expandafter\gdef\csname myarray(#1)\endcsname}
\def\dispsymbol#1{\csname myarray(#1)\endcsname} 


%START!!
%Titles
\title{Exploration of energy \\efficient memory \\organizations exploiting \\data variable based \\system scenarios}
%\subtitle{Subtitle of the thesis here}
\author{Iason Filippopoulos}

\def \thesisAuthor{Iason Filippopoulos}

%\degreetype{PhD}
\faculty{Faculty of Information Technology, Mathematics and Electrical Engineering}
\department{Department of Electronics and Telecommunications}
\copyrightnotice{All rights reserved}
\isbnprinted{}
\isbnelectronic{}
\serialnumber{}
\setyear{2015}
\setmonth{September}

\begin{document}

\frontmatter
\maketitle

%<<byman
%% Configuration of the header strings for the frontmatter pages.
\fancyhead[RO]{{\footnotesize\rightmark}\hspace{1em}\thepage}
\setcounter{tocdepth}{2}
\fancyhead[LE]{\thepage\hspace{1em}\footnotesize{\leftmark}}
\fancyhead[RE,LO]{}
\fancyhead[RO]{{\footnotesize\rightmark}\hspace{1em}\thepage}
%byman>>

%\vspace*{\fill}
%\addcontentsline{toc}{section}{Abstract}
\section*{\hspace*{\fill} Abstract \hspace*{\fill}}

Modern embedded systems are capable of performing a wide range of tasks and their popularity is increasing in many different application domains.
The recent progress in semiconductor processing technology greatly improves the performance of the embedded systems, due to the increased number of transistors on a single chip.
The continuous performance improvement, for example higher clock frequencies and lower supply voltage, increases the possibilities for new embedded system designs.
However, many embedded systems rely on a battery source, which puts a significant limitation on their lifetime and usage.
The management of the energy consumption is a key factor towards increasing the lifetime of an embedded system relying on a battery source.
%%Assuming that the development in the battery technology will follow the current trend, embedded systems should improve their energy efficiency based on the system design.
A typical embedded system includes one or more processing elements (CPUs, GPUs, embedded processors etc.), a memory subsystem (cache, scratchpad, flash, etc.) and application specific hardware (antennas, sensors, etc.). 
The memory subsystem contributes significantly  to the overall energy consumption as shown in many studies of  embedded system applications and platforms.
Especially for applications that are data intensive, the memory architecture may have the highest energy footprint of the whole embedded system.

%In this thesis, we focus on the design of energy efficient memory architectures for embedded systems.
A hardware/software co-design methodology is proposed for the reduction of the energy consumption in the memory subsystem based on the system scenario methodology.
In general, system scenario methodologies propose the use of different platform configurations in order to exploit run-time variations in application needs.
The current methodology exploits variations in memory needs during the lifetime of an application in order to optimize energy usage. 
The different resource requirements, that change dynamically at run-time, are grouped into scenarios to efficiently handle a very large exploration space.
Apart from the development of the methodology, an extended memory model is included in this work. 
The memory models is based on existing state-of-the-art memories, available from industry and academia.
In addition, the impact of the technology scaling is studied and the effectiveness of the proposed methodology is analyzed for the future memory architectures.
We also investigate the combination of the developed methodology with known code transformation techniques, specifically data interleaving.
The proposed design methodology aims at being compatible with the already available code optimization techniques.
Actual decrease in memory energy consumption for a selection of studied applications ranges between 30\% and 60\%. 
%We further extend the evaluation of the memory design methodology using a test-case wireless system.
%The proposed reconfigurable memory subsystem is studied in a dynamic platform with several reconfiguration options that combine the memory and the processing elements.
%\vspace*{\fill}
\afterpage{\null\newpage}
\newpage


\vspace*{\fill}
%\addcontentsline{toc}{section}{Preface}
\section*{\Huge Preface }
\bigskip
\bigskip
This doctoral thesis was submitted to the Norwegian University of Science and Technology (NTNU) in partial fulfillment of the requirements for the degree philosophiae doctor (PhD). 
The thesis is a part of a dual PhD program between NTNU and the Catholic University of Leuven (K. U. Leuven), in cooperation with Interuniversity Microelectronics Center (IMEC) in Leuven, Belgium, and Eindhoven, The Netherlands..
The work herein was performed at the Department of Electronics and Telecommunications, NTNU and the Department of Electrical Engineering, KU Leuven.
The work was performed under the supervision of Professor Per Gunnar Kjeldsberg and Professor Francky Catthoor.

\bigskip
\bigskip

%\addcontentsline{toc}{section}{Acknowledgements}
\section*{Acknowledgements}

\bigskip

I would like to thank my supervisors Prof. Per Gunnar Kjeldsberg and Prof. Francky Catthoor for their support and advise through the long process that eventually became this thesis. 
I also extend my gratitude to my co-supervisor Prof. Sverre Hendseth for providing encouragement and positive attitude all these years.

I would also like to thank Mladen for sharing both his direct and thought-provoking opinions and office space with me.
Furthermore, I thank Elena and Yahya for our great co-operation and their patience on the evaluation meetings.
I was lucky to meet great co-workers at NTNU and IMEC that offered their help. 

Finally, I would like to thank my family and friends for their support. 

\bigskip

\begin{flushright}
Iason Filippopoulos \\
September 2015
\end{flushright}

\vspace*{\fill}

%TOC
\tableofcontents
%\addcontentsline{toc}{chapter}{Contents}
\cleardoublepage

\listoftables	
\addcontentsline{toc}{chapter}{List of Tables}
\cleardoublepage

\listoffigures	
\addcontentsline{toc}{chapter}{List of Figures}
\cleardoublepage

%Abbreviations
\newpage
%\thispagestyle{empty}
\printnomenclature
\cleardoublepage

%Symbols
\newpage
\chapter*{List of Symbols\hfill} 
\addcontentsline{toc}{chapter}{List of Symbols}
\begin{flushleft}
%	\input{symbols}
\end{flushleft}

\mainmatter

\cleardoublepage
%Chapters
\chapter{Introduction}
\label{intro}


\section{Embedded Systems and Energy Consumption}

Embedded systems are usually designed to perform a specific tasks and often consist of domain-specific hardware.
For example, typical embedded systems use optimized processing cores to perform signal processing instead of using general purpose CPUs.
Some examples of embedded systems include TV sets, cellular phones, MP3 players, smart cameras, wireless access points and printers. 
An embedded system is designed with strong requirements regarding size, performance and power consumption.
The market demand is towards smaller and lighter devices.
The ever-progressing semiconductor processing technique has dramatically increased the number of transistors on a single chip, which makes today's hardware increasingly powerful.
Embedded systems often rely on a battery source to deliver the desired performance and the energy efficiency is a significant design factor.
Assuming that the development in the battery technology will follow the current trend, embedded systems should improve their energy efficiency based on the system design.

Memory subsystem has to meet the same requirements regarding size, performance and power consumption.
Many applications focusing on embedded systems are data intensive and the contribution of the memory to the overall system is significant.
This work focus on the exploration of energy efficient memory organizations suitable for embedded systems.
The goal is to provide a systematic way of designing a memory architecture that is energy efficient and meets performance requirements.
The current work presents a methodology to exploit variations in memory needs during the lifetime of an application in order to optimize energy usage.

\section{Dynamic Data Intensive Applications}

Data intensive applications perform tasks that involve operations on large sets of data.
Thus, the memory requirements of data intensive applications are high and the contribution of the memory important.
The main focus of this thesis is on applications that are both dynamic and data intensive.
The dynamism in this context refers to the significant changes in the behavior of the application.
In more detail, the studied applications exhibit a dynamic variation in the memory requirements during their lifetime.
The dynamic variation in the memory requirements can be input driver, which means that there is a wide variation on the execution of the application based on different inputs.
Because of this behavior, a static study of the application code alone is insufficient since the targeted applications have non-deterministic behavior that is driven by input.

\section{Problem Statement}

The general problem is expressed in the following form:
\begin{quote}
Given an embedded system application and its range of inputs, find the most suitable memory architecture and fully exploit its features to fulfill the performance requirements and reduce the energy consumption. 
\end{quote} 

The application is dynamic and the memory requirements vary through its lifetime, so there are opportunities for system optimization based on estimations on the system resources..
In order to provide performance guaranties the estimations should be pessimistic, and not optimistic, as over-estimations are acceptable, but under-estimations are generally not.
Currently used design approaches often use worst case estimations, which are obtained by statically analyzing the application. 
However, these techniques are not efficient when focusing on dynamic and input driven applications.
Due to the dynamism in target applications, the ratio of the worst case load versus the average load on the memory is normally high.
Hence, if only the worst case estimations are used during design, the resulting system would not be able to exploit this gap. 

A way to solve this problem is to design the system to meet the worst case requirements, but add reconfiguration knobs that can exploit the variation in the memory requirements (e.g., by switching off hardware components, which decreases the energy consumption).
A run-time mechanism that predicts the current application needs in term of resources and exploits this information should be also integrated into the system.
To enable this exploitation, the possible run-time situations (RTS) \nomenclature{RTS}{run-time situation} in which the application may run, together with their resource needs should be known and taken into account during design. 
The number of different inputs and the variations in the memory requirements for each RTS provide a huge exploration space that is difficult to handle, as it is almost impossible to enumerate every possible case.
Even if the explosion problem could be solved, it will be very difficult to predict at run-time in which RTS the application is running and the platform reconfiguration needed to better exploit the current RTS. 
In addition, the run-time overhead for switching to a different reconfiguration for every RTS could not be  compensated from the improvements in the energy consumption, because the reconfiguration of the platform has an energy penalty. 

\section{Current approaches and problems}

The presented problem has been studied before and different ways of tackling it have been proposed.
However, there are some aspects that have not been addressed before and are presented in this thesis.

Most of the current approaches rely on a static analysis of the target application and several methodologies have been presented to generate a static application-specific memory hierarchy \cite{Ben00b}.
Several techniques for designing energy efficient memory architectures for embedded systems are presented in \cite{Mac02}. 
The main limitation on these methodologies is the fact that they are applicable to applications with very limited dynamism. 
This work extends the state of the art by proposing a more generic approach, which is also suitable for applications with input driven dynamic behavior.  
In addition, the current work differentiates by employing a platform that is reconfigurable at run-time.
 
The approach of a reconfigurable memory platform has also been proposed several times and an extensive overview of current approaches is found in \cite{Garcia}.
Most of the proposed solutions are focusing on tackling one specific case-study application, which is divided in a small number of different cases based on observations at the user level.
However, our work differentiates by proposing a more generic and application agnostic methodology and analysis on the system level.
Thus, it can efficiently handle a wider range of dynamic application characteristics.

Another proposed approach to tackle the problem, is to focus on source code transformations, and especially loop transformations.
These methods try to modify the application code and provide an improved version of the application with easier memory management.
The main drawback of the code transformation approach is that it is not always possible to achieve the desired behavior, because the applications can be complex.
In any case, these methods are fully complementary to the methodology presented in this thesis and should be performed as a prior step to the current work. 

\section{Thesis contributions}

In this thesis, we focus on the design of energy efficient memory architectures for embedded systems.
A hardware/software co-design methodology is proposed for the reduction of the energy consumption on the memory subsystem. 
The  methodology exploits variations in memory needs during the lifetime of an application in order to optimize energy usage. 
The different resource requirements, that change dynamically at run-time, are organized into groups to efficiently handle a very large exploration space.
Apart from the development of the methodology, an extended memory model is included in this work. 
The memory models is based on existing state-of-the-art memories, available from industry and academia.

In addition, the impact of the technology scaling is studied and the effectiveness of the proposed methodology is analyzed for the future memory architectures.
We also investigate the combination of the developed methodology with known code transformation techniques, specifically data interleaving.
The proposed design methodology aims to be compatible with the already available code optimization techniques.
We further extend the evaluation of the memory design methodology using a test-case wireless system.
The proposed reconfigurable memory subsystem is studied in a dynamic platform with several reconfiguration options that combine the memory and the processing elements.

\section{Thesis Outline}

The remainder of this thesis is organized as follows: Chapter 2 contains background information regarding the work performed in the arrays of data transformations and system scenarios. 
Chapter 3 describes the research process and presents the developed methodology.
Each paper is described in chapter 4 with a breakdown of the roles of each author. 
Chapter 5 concludes the thesis with a summary of contributions. 
The appendix holds each paper I have authored or coauthored in chronological order. 
These papers are reproduced faithfully with regard to the published text, but has been reformatted to increase readability.
 
\chapter{Background} %use enough space up to 15-20 pages
\label{background}

\section{Data and Memory Management Approaches}
Literature other than DTSE...

\section{Data Transfer and Storage Exploration}
DTSE literature...

\section{System Scenarios}
All the literature with focus on the ones related

\subsection{Use-Case vs. System Scenarios}

Scenario based design has been used for a long time in different areas, like human-computer interaction or object oriented software engineering. 
Examples...

Use case scenario approaches generate different scenarios based on a user's behavior.
Examples...

System scenario methodology focuses on the behavior of the system to generate scenarios and can, therefore, fully exploit the detailed platform mapping information. 
Examples...
%In both these cases, these scenarios concretely describe, in an early phase of the development process, the use of a future system. In case of human-computer interaction, the scenarios appear like narrative descriptions of envisioned usage episodes, and in case of object oriented software engineering like a unified modeling language (UML) use-case diagram which enumerates, from functional and timing point of view, all possible user actions and the system reactions that are required to meet a proposed system function. These scenarios are called use-case scenarios.
%In the embedded systems area, use-case scenarios are used in both hardware [52, 85] and software design [29]. In these cases, the scenarios focus on the application’s functional and timing behaviors and on its interaction with the users and environment, not on the resources required by a system to meet its constraints. These scenarios are used as an input during system design for user- centered design approaches.
%This thesis concentrates on a different type of scenarios, so-called application scenarios, which may be derived from the behavior of the application. These scenarios are used to reduce the system cost by exploiting information about what can happen at runtime to make better design decisions. While use-case scenarios classify the application’s behavior based on the different ways it can be used, application scenarios classify it from the resource usage perspective, based on the cost trade-off aspects during the mapping to the platform. This second type of scenarios was for the first time explicitly identified and exploited by researchers from IMEC, Belgium, in [119].
%Figure 2.1 depicts a design trajectory using use-case and application scenarios. It starts from a product idea, for which the stakeholders1 manually define the product’s functionality as use-case scenarios. These scenarios characterize the system from a user perspective and are used as an input to the design of an embedded system that includes both software and hardware components. In order to optimize the design of the system, the detection and usage of application scenarios augments this trajectory (the bottom gray box in the figure). Once the application is coded, its scenarios related to resource utilization are extracted in an automatic way, and they are considered for the decisions made during the following phases of the system design. Hence, the runtime behavior of the application is classified into several application scenarios, where the cost of the operation modes within a scenario is always fairly similar. For each individual scenario, more specific and aggressive design decisions can be made.
%The sets of use-case scenarios and application scenarios are not necessarily disjoint, and it is possible that one or more use-case scenarios correspond to one application scenario. But still, usually they are not overlapping and it is likely that a use-case scenario is split into several application scenarios, or that several application scenarios intersect several use-case scenarios.


\section{Scratch-pad Memory Architectures - related work incl}





\chapter{Solution Approach}
\label{method}
%make a consistent story about how the papers are connetcted

\section{Target platform architecture}

\subsection{Target memory platform architecture}

\subsection{Memory models}

\subsection{Technology scaling}

\section{Data variable based memory-aware system scenario methodology}

\subsection{Design-time profiling based on data variables}

\subsection{Design-time system scenario identification based on data variables}

\subsection{Run-time system scenario detection and switching based on data variables}

\subsection{Interleaving exploration based on data variables}


\chapter{Research Results and Contributions}
\label{research}

\section{Contribution A: Memory-Aware Methodology Development}

\section{Contribution B: System Scenarios on Memory and PEs}

\section{Contribution C: Integrated Interleaving and Data-to-Memory Mapping}

\section{Contribution D: Interconnection Cost Modeling and Scaling}

\section{Paper A.I - Abstract - My contribution}

\section{Paper A.II}

\section{Paper A.III}

\section{Paper B.I}

\section{Paper C.I}

\section{Paper D.I}
\input{conclusions}    


\end{document}
% arara: pdflatex: {synctex: yes, action: nonstopmode}
% arara: pdflatex 
% arara: bibtex 
% arara: pdflatex
% arara: pdflatex
% arara: nomencl
% arara: pdflatex: {synctex: yes, action: nonstopmode}

%\documentclass[a4crop]{ntnuthesis} 
\documentclass{ntnuthesis}  
\usepackage{amsmath} %equations and spacings
\usepackage{lipsum} 
\usepackage{graphicx}
\usepackage[small,bf]{caption}
\usepackage[labelformat=simple]{subcaption}
\renewcommand\thesubfigure{(\alph{subfigure})} % see subcaption doc

%spaces under sections
\usepackage[compact]{titlesec}
\titlespacing{\section}{0pt}{*0}{*0}
\titlespacing{\subsection}{0pt}{*0}{*0}
\titlespacing{\subsubsection}{0pt}{*0}{*0}

%Abbreviations or Acronyms
%\usepackage[intoc]{nomencl}
\usepackage{nomencl}
\renewcommand{\nomname}{List of Abbreviations}
\makenomenclature

%Bibliography
%\usepackage{url}
\usepackage{natbib}%\usepackage[sectionbib]{natbib}
\usepackage{chapterbib}
%\bibpunct[:]{(}{)}{;}{a}{}{,} %citation structure
\bibpunct{(}{)}{,}{a}{}{;} 
%\bibpunct{[}{]}{,}{a}{}{;}
%\bibpunct{(}{)}{;}{a}{}{,} % to follow the A&A style
%\usepackage{chapterbib}
\usepackage{hyperref}
%\hypersetup{colorlinks=true,citecolor=blue}
\hypersetup{colorlinks=true,citecolor=blue,linkcolor=blue,urlcolor=blue}

%To change box color around the links and citations, you have these other options :
%\hypersetup{citebordercolor=Violet,filebordercolor=Red,linkbordercolor=Blue}

\newcommand{\cmt}[1]{}%inline comment
  
%tabularx
\usepackage{tabularx}
\usepackage{tabulary}
\usepackage{longtable} %table on several pages
\usepackage{booktabs}
\usepackage{multirow}
\usepackage{dcolumn}

%graphics
\usepackage{epstopdf} %support for eps.
\DeclareGraphicsExtensions{.eps,.ps,.pdf,.png,.jpg}
\usepackage{float} % figure placing [H]

%landscape Option
\usepackage{lscape} %Left down
%usepackage{pdflscape} %Left up)

\usepackage[draft]{todonotes}   % notes showed (JJUNJU)
% Select what to do with command \comment:  
% \newcommand{\comment}[1]{}  %comment not showed
\newcommand{\comment}[1]
{\par {\bfseries \color{red} #1 \par}} %comment showed

%Section Numbering and TOC depth
\setcounter{secnumdepth}{2}
\setcounter{tocdepth}{3}

%chapter biblio
\usepackage{chapterbib}

%To generate list of symbols
\newcommand{\addsymbol}[3]{%
  \symboldisplay{#1}{#2}\\%
  \setelem{#3}{#1}
}
\newcommand{\symboldisplay}[2]{%
  $#1$ \parbox{5in}{\dotfill #2}%
  %$#1$ \parbox{5in}{ #2}%
}
%\def\setelem#1{\expandafter\def\csname myarray(#1)\endcsname}
\def\setelem#1{\expandafter\gdef\csname myarray(#1)\endcsname}
\def\dispsymbol#1{\csname myarray(#1)\endcsname} 

%START!!
%Titles
\title{Exploration of energy \\efficient memory \\organizations exploiting \\data variable based \\system scenarios}
%\subtitle{Subtitle of the thesis here}
\author{Iason Filippopoulos}

\def \thesisAuthor{Iason Filippopoulos}

%\degreetype{PhD}
\faculty{Faculty of Information Technology, Mathematics and Electrical Engineering}
\department{Department of Electronics and Telecommunications}
\copyrightnotice{All rights reserved}
\isbnprinted{}
\isbnelectronic{}
\serialnumber{}
\setyear{2015}
\setmonth{September}

\begin{document}

\frontmatter
\maketitle

%<<byman
%% Configuration of the header strings for the frontmatter pages.
\fancyhead[RO]{{\footnotesize\rightmark}\hspace{1em}\thepage}
\setcounter{tocdepth}{2}
\fancyhead[LE]{\thepage\hspace{1em}\footnotesize{\leftmark}}
\fancyhead[RE,LO]{}
\fancyhead[RO]{{\footnotesize\rightmark}\hspace{1em}\thepage}
%byman>>

\vspace*{\fill}
\addcontentsline{toc}{section}{Abstract}
\section*{\hspace*{\fill} Abstract \hspace*{\fill}}

Modern embedded systems are capable of performing a wide range of tasks and their popularity is increasing in many different application domains.
The recent progress on the semiconductor processing technology greatly improves the performance of the embedded systems, due to the increased number of transistors on a single chip.
The continuous performance improvement increases the possibilities for new embedded system designs.
However, many embedded systems rely on a battery source, which puts a significant limitation on their lifetime and usage.
The management of the energy consumption is the key factor towards increasing the lifetime of an embedded system relying on a battery source.
%%Assuming that the development in the battery technology will follow the current trend, embedded systems should improve their energy efficiency based on the system design.
A typical embedded system includes one or more processing elements (CPUs, GPUs, embedded processors etc.), a memory subsystem (cache, scratchpad, flash, etc.) and application specific hardware (antennas, sensors, etc.). 
The memory subsystem is a significant contribution to the overall energy based on the study of existing and proposed embedded systems.
Especially for applications that are data intensive, the memory architecture may have the highest energy footprint of the whole embedded system.

In this thesis, we focus on the design of energy efficient memory architectures for embedded systems.
A hardware/software co-design methodology is proposed for the reduction of the energy consumption on the memory subsystem. 
The  methodology exploits variations in memory needs during the lifetime of an application in order to optimize energy usage. 
The different resource requirements, that change dynamically at run-time, are organized into groups to efficiently handle a very large exploration space.
Apart from the development of the methodology, an extended memory model is included in this work. 
The memory models is based on existing state-of-the-art memories, available from industry and academia.
In addition, the impact of the technology scaling is studied and the effectiveness of the proposed methodology is analyzed for the future memory architectures.
We also investigate the combination of the developed methodology with known code transformation techniques, specifically data interleaving.
The proposed design methodology aims to be compatible with the already available code optimization techniques.
%We further extend the evaluation of the memory design methodology using a test-case wireless system.
%The proposed reconfigurable memory subsystem is studied in a dynamic platform with several reconfiguration options that combine the memory and the processing elements.
\vspace*{\fill}
\afterpage{\null\newpage}
\newpage


\vspace*{\fill}
\addcontentsline{toc}{section}{Preface}
\section*{\Huge Preface }
\bigskip
\bigskip
This doctoral thesis was submitted to the Norwegian University of Science and Technology (NTNU) in partial fulfillment of the requirements for the degree philosophiae doctor (PhD). 
The thesis is a part of a dual PhD program between NTNU and the Catholic University of Leuven (K. U. Leuven), in cooperation with Interuniversity Microelectronics Center (IMEC).
The work herein was performed at the Department of Electronics and Telecommunications, NTNU and the Department of Electrical Engineering, KU Leuven.
The work was performed under the supervision of Professor Per Gunnar Kjeldsberg and Professor Francky Catthoor.

\bigskip
\bigskip

\addcontentsline{toc}{section}{Acknowledgements}
\section*{Acknowledgements}

\bigskip

I would like to thank my supervisors Prof. Per Gunnar Kjeldsberg and Prof. Francky Catthoor for their support and advise through the long process that eventually became this thesis. 
I also extend my gratitude to my co-supervisor Prof. Sverre Hendseth for providing encouragement and positive attitude all these years.

I would also like to thank Mladen for sharing both his direct and thought-provoking opinions and office space with me.
Furthermore, I thank Elena and Yahya for our great co-operation and their patience on the evaluation meetings.
I was lucky to meet great co-workers at NTNU and IMEC that offered their help. 

Finally, I would like to thank my family and friends for their support. 

\bigskip

\begin{flushright}
Iason Filippopoulos \\
September 2015
\end{flushright}

\vspace*{\fill}

\addcontentsline{toc}{section}{Preface}
\section*{Preface}



\addcontentsline{toc}{section}{Acknowledgements}
\section*{Acknowledgements}



%TOC
\tableofcontents
%\addcontentsline{toc}{chapter}{Contents}
\cleardoublepage

\listoftables	
\addcontentsline{toc}{chapter}{List of Tables}
\cleardoublepage

\listoffigures	
\addcontentsline{toc}{chapter}{List of Figures}
\cleardoublepage

%Abbreviations
\newpage
%\thispagestyle{empty}
\printnomenclature
\cleardoublepage

%Symbols
\newpage
\chapter*{List of Symbols\hfill} 
\addcontentsline{toc}{chapter}{List of Symbols}
\begin{flushleft}
%	\input{symbols}
\end{flushleft}

\mainmatter

\cleardoublepage
%Chapters
\chapter{Introduction}
\label{intro}


\section{Embedded Systems and Energy Consumption - what is our general goal}

\section{Data Intensive Applications - what is our general goal}

\section{Brief summary of current way of tackling and what has not been addressed here}

\section{Problem Statement}

\section{Proposed Solution}

\section{Thesis Outline - what is covered in which paper} 
\chapter{Background} %use enough space up to 15-20 pages
\label{background}

\section{Scratch-pad Memory Architectures - related work incl}

\section{System Scenarios - all the literature with focus on the ones related}
	- vs. use case
	- pareto

\section{DTSE}
	-


\chapter{Methodology}
\label{method}
%make a consistent story about how the papers are connetcted

\section{System Scenario Platform}

\subsection{Target memory platform architecture}

\subsection{Memory models}

\subsection{Technology scaling}

\section{Data variable based memory-aware system scenario methodology}

\subsection{Interleaving exploration based on data variables}

\subsection{Design-time profiling based on data variables}

\subsection{Design-time system scenario identification based on data variables}

\subsection{Run-time system scenario detection and switching based on data variables}




\chapter{Research Results and Contributions}
\label{research}

\section{Contribution A: Memory-Aware Methodology Development}

\section{Contribution B: System Scenarios on Memory and PEs}

\section{Contribution C: Integrated Interleaving and Data-to-Memory Mapping}

\section{Contribution D: Interconnection Cost Modeling and Scaling}

\section{Paper A.I - Abstract - My contribution}

\section{Paper A.II}

\section{Paper A.III}

\section{Paper B.I}

\section{Paper C.I}

\section{Paper D.I}
\chapter{Conclusions}
\label{conclusions}

    


\end{document}
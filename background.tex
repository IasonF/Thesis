\chapter{Background} %use enough space up to 15-20 pages
\label{background}

\section{Data and Memory Management Approaches}
Literature other than DTSE...

\section{Data Transfer and Storage Exploration}
DTSE literature...

\section{System Scenarios}
All the literature with focus on the ones related

\subsection{Use-Case vs. System Scenarios}

Scenario based design has been used for a long time in different areas, like human-computer interaction or object oriented software engineering. 
Examples...

Use case scenario approaches generate different scenarios based on a user's behavior.
Examples...

System scenario methodology focuses on the behavior of the system to generate scenarios and can, therefore, fully exploit the detailed platform mapping information. 
Examples...
%In both these cases, these scenarios concretely describe, in an early phase of the development process, the use of a future system. In case of human-computer interaction, the scenarios appear like narrative descriptions of envisioned usage episodes, and in case of object oriented software engineering like a unified modeling language (UML) use-case diagram which enumerates, from functional and timing point of view, all possible user actions and the system reactions that are required to meet a proposed system function. These scenarios are called use-case scenarios.
%In the embedded systems area, use-case scenarios are used in both hardware [52, 85] and software design [29]. In these cases, the scenarios focus on the application’s functional and timing behaviors and on its interaction with the users and environment, not on the resources required by a system to meet its constraints. These scenarios are used as an input during system design for user- centered design approaches.
%This thesis concentrates on a different type of scenarios, so-called application scenarios, which may be derived from the behavior of the application. These scenarios are used to reduce the system cost by exploiting information about what can happen at runtime to make better design decisions. While use-case scenarios classify the application’s behavior based on the different ways it can be used, application scenarios classify it from the resource usage perspective, based on the cost trade-off aspects during the mapping to the platform. This second type of scenarios was for the first time explicitly identified and exploited by researchers from IMEC, Belgium, in [119].
%Figure 2.1 depicts a design trajectory using use-case and application scenarios. It starts from a product idea, for which the stakeholders1 manually define the product’s functionality as use-case scenarios. These scenarios characterize the system from a user perspective and are used as an input to the design of an embedded system that includes both software and hardware components. In order to optimize the design of the system, the detection and usage of application scenarios augments this trajectory (the bottom gray box in the figure). Once the application is coded, its scenarios related to resource utilization are extracted in an automatic way, and they are considered for the decisions made during the following phases of the system design. Hence, the runtime behavior of the application is classified into several application scenarios, where the cost of the operation modes within a scenario is always fairly similar. For each individual scenario, more specific and aggressive design decisions can be made.
%The sets of use-case scenarios and application scenarios are not necessarily disjoint, and it is possible that one or more use-case scenarios correspond to one application scenario. But still, usually they are not overlapping and it is likely that a use-case scenario is split into several application scenarios, or that several application scenarios intersect several use-case scenarios.


\section{Scratch-pad Memory Architectures - related work incl}





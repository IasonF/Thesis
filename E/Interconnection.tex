\documentclass[12pt,a4paper]{article}
\usepackage[latin1]{inputenc}
\usepackage{amsmath}
\usepackage{amsfonts}
\usepackage{amssymb}
\usepackage{graphicx}
\author{Iason Filippopoulos \and Francky Catthoor \and Per Gunnar Kjeldsberg}
\title{Technology scaling impact on the interconnect of clustered scratchpad memory architectures}

\begin{document}
\maketitle

\begin{abstract}
This work
\end{abstract}

\section{Introduction}


\section{Related Work}


\section{Technology Scaling}

\subsection{Memory Banks}

\subsection{Interconnection}


\section{Example design synthesis and simulation}

\subsection{Generic Work-flow}

Interconnection models are work-in-progress with the help of Jan. There are no results yet, but the following procedure is followed:

\begin{itemize}
	\item Choose a number of memories compiled by Jan (currently two)
	\item Create RTL description of the design connecting the memories using MUX, signals etc. (Done for two memories)
	\item Simulate to see if it works (Done for two memories)
	\item Select target technology and run logic synthesis (wip)
	\item Run floorplanning and place \& route (wip)
	\item Extract parasitics 
	\item Run static timing analysis and create SDF file 
	\item Annotate timing to the netlist	
	\item Run dynamic timing and power simulation
\end{itemize}

I still work on getting results for the first design that connects two memories. Then, I could estimate the time needed for making every new design. I would like to discuss with you how we should build a more generic model from those results, as it would not be possible to make the procedure outside IMEC for future configurations.

\begin{enumerate}
\item RTL description of the design (Done)
\item Simulation and verification of correct operation (Done)
\item Logic synthesis on the target technology (Done)
\item Floor-planning (Done)
\item Place and route (Errors) 
\item Timing and power analysis (Not done)
\end{enumerate}

\begin{itemize}
\item An example of two memories is shown on the figure below.
\item Dimensions for all SCMEMs are available from the floorplanning step.
\item Number of wires for each design are available.
\item Capacitance values for the wires on the figure are available. 
\end{itemize}

 
\textbf{Model:}

\begin{itemize}
\item Given the dimensions of the memories an estimation about length of the needed wires can be made.
\item Capacitance for the wires can be approximated by the results given for the studied example.
\item Power can be calculated as $Power = \dfrac{1}{2} \times f \times C \times V_{dd}^{2} $
\end{itemize}


\section{Model Construction and Projection Results}


\section{Conclusion}


\end{document}





%%OLD....
%\RequirePackage{fix-cm}
%%\documentclass[smallextended]{svjour3}
%\documentclass[smallcondensed]{svjour3}  
%\smartqed  % flush right qed marks, e.g. at end of proof
%
%\usepackage{graphicx}
%\usepackage{amsfonts}
%\usepackage{amssymb}
%\usepackage{array}
%\usepackage{amsmath}
%\usepackage{multirow}
%\usepackage{algorithmic}
%\usepackage{algorithm}
%
%\begin{document}
%
%\title{Technology scaling impact on the interconnect of clustered scratchpad memory architectures}
%
%\titlerunning{Technology scaling impact on the interconnect} 
%
%\author{Iason Filippopoulos \and Francky Catthoor \and Per Gunnar Kjeldsberg}
%
%\authorrunning{Filippopoulos et al.} % if too long for running head
%
%\institute{Iason Filippopoulos, Per Gunnar Kjeldsberg 
%		\at Department of Electronics and Telecommunications \\ Norwegian University of Science and Technology (NTNU), Norway\\
%		\email{iason.filippopoulos, pgk@iet.ntnu.no} \\
%		\and
%		Francky Catthoor 
%		\at IMEC \& KU Leuven, Belgium \\
%		\email{catthoor@imec.be} \\
%}
%
%\date{Received: date / Accepted: date}
%% The correct dates will be entered by the editor
%
%
%\maketitle
%
%\begin{abstract}
%We propose a memory-aware system scenario approach that exploits variations in memory needs during the lifetime of an application in order to optimize energy usage. 
%Different system scenarios capture the application's different resource requirements that change dynamically at run-time. 
%In addition to computational resources, the many possible memory platform configurations and data-to-memory assignments are important system scenario parameters. 
%In this work we focus on clustering of different memory requirements into groups and presenting the system scenario generation in detail.
%The clustering is a non-trivial problem due to the many different memory requirements, which leads to a very large exploration space.
%An extended memory model is used as a practical enabler, in order to evaluate the methodology. 
%\keywords{System Scenarios \and Design Space Exploration \and Reconfigurable Design \and Memory Reconfiguration \and Dynamic Multimedia Applications} 
%\end{abstract}
%
%
%\section{Introduction}
%\label{sec:introduction}
%
%Modern embedded systems are becoming more and more powerful as the semiconductor processing techniques keep increasing the number of transistors on a single chip. 
%Consequently, demanding applications, e.g., in the signal processing and multimedia domains, can be executed on these devices \cite{narasinga}. 
%On the other hand, the desired performance has to be delivered with minimum power consumption due to the limited energy available in mobile devices \cite{tcm}. 
%System scenario methodologies propose the use of different platform configurations in order to exploit run-time variations in computational and memory needs often seen in such applications \cite{tcm}.
%
%Platform reconfiguration is performed through tuning of different system parameters, also called system knobs. 
%For the memory-aware system scenario methodology, a platform can be reconfigured through a number of potential knobs, each resulting in different performance and power consumption in the memory subsystem. 
%Foremost, modern memories support different energy states, e.g., through power gating techniques and by switching to lower power modes when not accessed. 
%The second platform knob is the assignment of data to the available memory banks.
%The data assignment decisions affect both the energy per access for the mapped data, the data conflicts as a result of suboptimal assignment, and the number of active banks. 
%In this work a reconfigurable memory platform is constructed using detailed memory models. 
%This is followed by experiments with dynamic multimedia applications in order to study the effectiveness of the methodology.
%
%The main contribution of the current work is the development of data variable \cite{Elena2012} based system scenarios.
%Previous control variable based system scenarios \cite{Gheorghita2007} are unable to handle the fine-grain behaviour of the studied multimedia applications due to their significant variation under different execution situations. 
%Furthermore, compared with use case scenario approaches in which scenarios are generated based on a user's behaviour \cite{usecase}, the system scenario methodology focuses on the behaviour of the system to generate scenarios and can, therefore, fully exploit the detailed platform mapping information. 
%Compared with previous work on system scenarios that has focused on the processing cores, the current work analyses the use of system scenarios on the memory organisation. 
%More specifically, this work focuses on the system scenario identification phase of the methodology.
%The wide range of memory requirements, the amount of different cases, and the different frequency in which each case occurs, results in a very large exploration space.
%Therefore, there is a need for developing an algorithmic approach that can efficiently tackle this problem.
%
%Another significant contribution is the extensive number of benchmark applications on which the methodology is applied.
%The chosen set is representative for the domain of multimedia applications.
%Furthermore, we present a categorisation of applications based on their dynamic characteristics, also applicable to the entire multimedia domain. 
%For the experimental needs of this work we present for the purpose sufficiently detailed and accurate  memory models, which are used for the system design exploration.
%For the multimedia domain, the current work presents a comprehensive methodology for optimising energy consumption in the memory subsystem.
%
%This article is organized as follows. 
%Section~\ref{sec:motivation} motivates the study of optimization of the memory organisation. 
%Section~\ref{sec:related} surveys related work on system level memory exploration and on system scenario methodologies and compares it with the current work. 
%Section~\ref{sec:methodology} presents the chosen methodology with main focus on the memory organisation study. 
%In Section~\ref{sec:platform} the target platform is described accompanied by a detailed description of the employed memory models, while the multimedia benchmarks and their characteristics are analysed in Section~\ref{sec:applications}. 
%Results of applying the described methodology to the targeted applications are shown in Section~\ref{sec:results}, while conclusions are drawn in Section~\ref{sec:conclusion}. 
%
%\section{Motivational Example}
%\label{sec:motivation}
%
%\section{Related Work and Contribution Discussion}
%\label{sec:related}
%
%\section{Conclusions}
%\label{sec:conclusion}
%
%The scope of this work is to apply the memory-aware system scenario methodology to a wide range of multimedia application and test its effectiveness based on an extensive memory energy model. 
%A wide range of applications is studied that allow us to draw conclusions about different kinds of dynamic behaviour and their effect on the energy gains achieved using the methodology. 
%The results demonstrate the effectiveness of the methodology reducing the memory energy consumption with between 35\% and 55\%. 
%Since memory size requirements are still met in all situations, performance is not reduced. 
%The memory-aware system scenario methodology is suited for applications that experience dynamic behaviour with respect to memory organisation utilization during their execution.
%
%
%
%\end{document}
